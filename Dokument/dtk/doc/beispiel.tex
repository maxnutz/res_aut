\documentclass[ngerman]{dtk}
\ifluatex\else
  \usepackage[utf8]{inputenc}
  %\usepackage[latin9]{inputenc}
\fi

\addbibresource{beispiel.bib}

\let\File\texttt
\let\Package\texttt

\begin{document}
\title{Die \texttt{dtk}-Klasse, Version \DTKversion -- \DTKversiondate}
\Author{Mustermann}{Max}%
    {Dorfstraße~1\\
     14129 Berlin\\
     \Email{Max.Mustermann@xyz.de}}
\Author{Peter}{Silie}%
    {Hauptstr.~2\\
     10245~Berlin\\
     \Email{Peter.Silie@abc.de}}
\maketitle

\section{Makros}

\begin{verbatim}
\DeclareUrlCommand\File{\urlstyle{tt}}
\providecommand\Package[1]{\texttt{#1}}
\providecommand\Program[1]{\texttt{#1}}
\providecommand\Command[1]{\texttt{#1}}
\providecommand\Macro[1]{\texttt{\#1}}
\providecommand\Environment[1]{\texttt{#1}}
\let\Email\url
\end{verbatim}

Die Dateien \File{*.clo} beinhalten die Klassenoptionen und
werden nur fortgeschrieben, wenn etwas Neues einzufügen ist. Das Paket
\Package{dtk} sollte nur um wichtige Pakete ergänzt werden, die für den Inhalt
des Manuskripts wichtig sind. Das Programm \Program{lualatex} sollte bevorzugt
zum Übersetzen eingesetzt werden. Macros wie \Macro{texttt} müssen nicht mit
dem Backslash angegeben werden, der wird automatisch gesetzt. Es folgen
Beispiele mit Listings in nummerierter und in nichtnummerierter Art:

\begin{lstlisting}[style=number]
\begin{itemize}
\item ...
\end{itemize}
\end{lstlisting}

\begin{lstlisting}[style=noNumber]
\begin{description}
\item ...
\end{description}
\end{lstlisting}

\textsf{Ubuntu} wird in serifenloser Schrift gesetzt.
Weitere Informationen für Autoren findet man im DTK-Wiki:
\url{http://projekte.dante.de/DanteFAQ/WebHome}.


\section{Listen}
\begin{itemize}
\item Bei \Environment{itemize} werden
die Elemente durch Punkte und andere Symbole gekennzeichnet. 
\item Listen kann man auch verschachteln:
  \begin{itemize}
  \item Die maximale Schachtelungstiefe ist~4.~\cite{voss:2012}
  \item
  Bezeichnung und Ein\-rückung der Elemente
  wechseln automatisch.
  \end{itemize}
\item usw.
\end{itemize}

\begin{enumerate}
\item Bei \Environment{enumerate} werden
die Elemente mit Ziffern oder Buchstaben numeriert.
\item Die Numerierung erfolgt automatisch.~\cite{pakin:2008}
\item Listen kann man auch
verschachteln:
  \begin{enumerate}
  \item Die maximale Schachtelungstiefe
  ist~4.
  \item Bezeichnung und Ein\-rückung der Elemente
  wechseln automatisch.
  \end{enumerate}
\item usw.
\end{enumerate}

\begin{description}
\item[Gelse:]
   ein kleines Tier, das
   östlich des Semmering Touristen verjagt.
\item[Gemse:]
   ein großes Tier, das
   westlich des Semmering von Touristen verjagt wird.
\item[G"urteltier:]
   ein mittelgroßes Tier, das
   hier nur wegen der Länge seines Namens vorkommt.
\end{description}

\section{Bibliografie}
Die folgende Bibliografie hat nur einen 
Demonstrationscharakter.

\begin{lstlisting}[style=number]
\printbibliography
\end{lstlisting}

\nocite{*}
\printbibliography


\section{Quelltext}
Zm Schluss folgt der Quelltext dieses Beispieldokumentes:
\lstinputlisting[style=number,language={[AlLaTeX]{TeX}}]{\jobname.tex}


\end{document}

